% Manual de Usuario - Sistema de Evaluación Docente
% Basado en la norma IEEE 1063-2001

\documentclass[12pt,a4paper]{article}
\usepackage[utf8]{inputenc}
\usepackage[spanish]{babel}
\usepackage{geometry}
\usepackage{graphicx}
\usepackage{hyperref}
\usepackage{titlesec}
\usepackage{tocloft}
\usepackage{xcolor}
\usepackage{enumitem}
\usepackage{fancyhdr}
\usepackage{booktabs}
\usepackage{tabularx}
\usepackage{float}
\usepackage{tikz}
\usetikzlibrary{shapes.geometric, arrows.meta, positioning}


% Configuración del documento
\geometry{margin=2.5cm}
\hypersetup{
    colorlinks=true,
    linkcolor=blue,
    filecolor=magenta,
    urlcolor=blue,
}

% Configuración de encabezados y pies de página
\pagestyle{fancy}
\fancyhf{}
\renewcommand{\headrulewidth}{0.4pt}
\renewcommand{\footrulewidth}{0.4pt}
\fancyhead[L]{Sistema de Evaluación Docente}
\fancyhead[R]{IEEE 1063}
\fancyfoot[C]{\thepage}

% Configuración para títulos
\titleformat{\section}
  {\normalfont\Large\bfseries}{\thesection}{1em}{}
\titleformat{\subsection}
  {\normalfont\large\bfseries}{\thesubsection}{1em}{}

% Documento
\begin{document}

% Portada mejorada conforme a IEEE 1063-2001
\begin{titlepage}
\begin{center}
\vspace*{2cm}
{\Huge \textbf{Manual de Usuario}}\\[0.5cm]
{\LARGE \textbf{Sistema de Evaluación Docente}}\\[0.5cm]
{\large Conforme a la norma IEEE 1063-2001}\\[2cm]
\includegraphics[width=0.45\textwidth]{images/Logo Uniautonoma.png}\\[1cm]
\textbf{Versión del Sistema:} 1.0\\
\textbf{Versión del Documento:} 1.0\\
\textbf{Fecha de Emisión:} \today\\[2.5cm]
\textbf{Desarrollado por:}\\[0.3cm]
Thomas Montoya Magon\\
Juan Daniel Bravo\\
Alejandro Martínez Salazar\\
Daniel Rivas Agredo\\
Luisa Julieth Joaqui\\[2cm]
\textbf{Contacto:} soporte@evaluaciondocente.edu.co\\
\vfill
\textit{Este documento proporciona instrucciones para el uso adecuado del sistema y está dirigido a todos los roles autorizados.}\\
\end{center}
\end{titlepage}

%-------------------------------
% Historial de Revisiones
%-------------------------------
\clearpage
\section*{Historial de Revisiones}
\addcontentsline{toc}{section}{Historial de Revisiones}
\begin{center}
\begin{tabular}{lll}
\toprule
\textbf{Versión} & \textbf{Fecha} & \textbf{Descripción} \\
\midrule
1.0 & \today & Creación inicial del manual conforme a IEEE 1063-2001 \\
\bottomrule
\end{tabular}
\end{center}

%-------------------------------
% Índices
%-------------------------------
\clearpage
\tableofcontents
\clearpage
\listoffigures
\addcontentsline{toc}{section}{Lista de Figuras}
\clearpage
\listoftables
\addcontentsline{toc}{section}{Lista de Tablas}
\clearpage

% 1. Introducción
\section{Introducción}\label{sec:introduccion}
\subsection{Propósito}
Este manual de usuario tiene como propósito proporcionar una guía clara y concisa sobre el uso del Sistema de Evaluación Docente, diseñado para optimizar el proceso de evaluación de docentes en instituciones educativas. El sistema permite la evaluación desde tres perspectivas diferentes: estudiantes, administrativos (decanos/coordinadores) y la autoevaluación del propio docente.

\subsection{Alcance}
Este documento está dirigido a todos los usuarios del sistema, incluyendo administradores, decanos, coordinadores y docentes. Proporciona instrucciones detalladas sobre cómo utilizar las diversas funcionalidades disponibles según el rol de cada usuario.

\subsection{Definiciones y Acrónimos}
\begin{table}[H]
  \centering
  \caption{Definiciones y Acrónimos}\label{tab:acronimos}
  \begin{tabularx}{\textwidth}{>{\bfseries}lX}
    \toprule
    Término & Definición \\
    \midrule
    SED & Sistema de Evaluación Docente \\
    Acta de Compromiso & Documento formal que registra los compromisos de mejora adquiridos por un docente \\
    Proceso de Sanción & Procedimiento disciplinario iniciado cuando un docente no cumple con los compromisos establecidos \\
    Dashboard & Panel de control personalizado con información relevante según el rol del usuario \\
    \bottomrule
  \end{tabularx}
\end{table}

% 2. Uso del documento
\section{Uso del Documento}\label{sec:uso-documento}
Este manual está organizado según la norma IEEE 1063 y se estructura en secciones que siguen el flujo lógico de utilización del sistema. Cada sección incluye descripciones detalladas e instrucciones paso a paso. Se recomienda leer primero la sección correspondiente a su rol en el sistema y luego explorar las funcionalidades específicas que necesite utilizar.

% 3. Conceptos de operación
\section{Conceptos de Operación}\label{sec:conceptos}
\subsection{Descripción General del Sistema}
El Sistema de Evaluación Docente es una aplicación web desarrollada en Laravel que permite:
\begin{itemize}
    \item Realizar evaluaciones de docentes por parte de estudiantes de forma anónima
    \item Gestionar evaluaciones administrativas por parte de coordinadores y decanos
    \item Facilitar la autoevaluación de los docentes
    \item Generar reportes, estadísticas y actas de compromiso
    \item Emitir alertas por bajo rendimiento
    \item Gestionar procesos de sanción cuando sea necesario
\end{itemize}

\subsection{Roles de Usuario}
El sistema contempla tres roles principales:
\begin{itemize}
    \item \textbf{Administrador:} Gestiona períodos de evaluación, roles y genera reportes globales.
    \item \textbf{Decano/Coordinador:} Gestiona actas de compromiso, monitorea alertas, realiza seguimiento a planes de mejora y gestiona procesos de sanción.
    \item \textbf{Docente:} Consulta resultados de evaluación, visualiza estadísticas personales y configura sus datos.
\end{itemize}

\subsection{Arquitectura del Sistema}
La aplicación sigue la arquitectura Modelo-Vista-Controlador (MVC) de Laravel, utilizando:
\begin{itemize}
    \item Frontend con Blade como motor de plantillas
    \item Backend en PHP con Laravel
    \item Base de datos MySQL
\end{itemize}

\begin{figure}[H]
  \centering
  \includegraphics[width=0.8\textwidth]{images/diagrama_mvc.png}
  \caption{Esquema MVC del sistema}
  \label{fig:mvc}
\end{figure}


% 4. Procedimientos
\section{Procedimientos}\label{sec:procedimientos}
\subsection{Acceso al Sistema}
\subsubsection{Inicio de Sesión}
\begin{enumerate}
    \item Abra su navegador web y acceda a la URL del sistema.
    \item En la página de inicio, encontrará un formulario de login.
    \item Ingrese su nombre de usuario y contraseña.
    \item Seleccione su rol (Administrador, Decano/Coordinador o Docente).
    \item Haga clic en el botón "Iniciar Sesión".
\end{enumerate}

\begin{figure}[H]
  \centering
  \includegraphics[width=0.6\textwidth]{images/LOGIN.png}
  \caption{Pantalla de inicio de sesión}
  \label{fig:login}
\end{figure}


\subsubsection{Cerrar Sesión}
\begin{enumerate}
    \item Haga clic en su nombre de usuario ubicado en la esquina superior derecha.
    \item Seleccione la opción "Cerrar Sesión".
\end{enumerate}

\subsection{Funcionalidades para Administradores}
\subsubsection{Acceso al Panel de Administrador}
\begin{enumerate}
    \item Inicie sesión con credenciales de administrador.
    \item Será redirigido automáticamente al panel de administrador.
    \item Alternativamente, puede acceder mediante la ruta: \texttt{/Admin}
\end{enumerate}

\subsubsection{Gestión de Períodos de Evaluación}
\begin{enumerate}
    \item En el panel de administrador, haga clic en "Períodos de Evaluación".
    \item Para crear un nuevo período, complete el formulario con:
    \begin{itemize}
        \item Nombre del período
        \item Fecha de inicio
        \item Fecha de finalización
        \item Estado (Activo/Inactivo)
    \end{itemize}
    \item Haga clic en "Guardar" para crear el período.
    \item Para editar un período existente, haga clic en el ícono de edición junto al período deseado.
    \item Para eliminar un período, haga clic en el ícono de eliminación.
\end{enumerate}

\begin{figure}[H]
  \centering
  \includegraphics[width=0.7\textwidth]{images/Periodo.png}
  \caption{Pantalla de gestión de períodos (Administrador)}
  \label{fig:admin_periodos}
\end{figure}

\begin{figure}[H]
  \centering
  \includegraphics[width=0.7\textwidth]{images/Periodo.png}
  \caption{Pantalla de gestión de períodos (Administrador)}
  \label{fig:admin_periodos}
\end{figure}

\subsubsection{Administración de Roles y Permisos}
\begin{enumerate}
    \item En el panel de administrador, haga clic en "Roles y Permisos".
    \item Para asignar un rol, seleccione el usuario y el rol correspondiente.
    \item Para modificar permisos, seleccione el rol y marque o desmarque los permisos deseados.
    \item Haga clic en "Guardar Cambios" para aplicar las modificaciones.
\end{enumerate}

\begin{figure}[H]
  \centering
  \includegraphics[width=0.7\textwidth]{images/Roles_y_Permisos.png}
  \caption{Pantalla de Administracion de Roles y Permisos (Administrador)}
  \label{fig:admin_periodos}
\end{figure}

\subsubsection{Generación de Reportes Globales}
\begin{enumerate}
    \item En el panel de administrador, haga clic en "Reportes".
    \item Seleccione el tipo de reporte que desea generar.
    \item Aplique filtros si es necesario (período, facultad, programa).
    \item Haga clic en "Generar Reporte".
    \item Utilice las opciones para exportar el reporte en formato PDF o Excel.
\end{enumerate}

\begin{figure}[H]
  \centering
  \includegraphics[width=0.7\textwidth]{images/Reportes_y_Estadisticas.png}
  \caption{Pantalla de Generacion de Reportes y Estadisticas(Administrador)}
  \label{fig:admin_periodos}
\end{figure}

\subsection{Funcionalidades para Decanos/Coordinadores}
\subsubsection{Acceso al Panel de Decano/Coordinador}
\begin{enumerate}
    \item Inicie sesión con credenciales de decano o coordinador.
    \item Será redirigido automáticamente al panel de decano.
    \item Alternativamente, puede acceder mediante la ruta: \texttt{/decano}
\end{enumerate}

\subsubsection{Gestión de Actas de Compromiso}
\begin{enumerate}
    \item En el panel de decano, haga clic en "Actas de Compromiso".
    \item Para crear una nueva acta, haga clic en "Nueva Acta".
    \item Complete el formulario con:
    \begin{itemize}
        \item Datos del docente
        \item Motivo del acta
        \item Compromisos establecidos
        \item Fecha de seguimiento
    \end{itemize}
    \item Haga clic en "Guardar Acta".
    \item Para editar un acta existente, haga clic en "Editar" junto al acta deseada.
    \item Para eliminar un acta, haga clic en "Eliminar".
\end{enumerate}

\begin{figure}[H]
  \centering
  \includegraphics[width=0.7\textwidth]{images/decano_actas.png}
  \caption{Gestión de actas de compromiso (Decano/Coordinador)}
  \label{fig:decano_actas}
\end{figure}


\subsubsection{Monitoreo de Alertas de Bajo Desempeño}
\begin{enumerate}
    \item En el panel de decano, haga clic en "Alertas de Bajo Desempeño".
    \item Visualizará un listado de docentes que han obtenido calificaciones por debajo del umbral establecido.
    \item Para ver detalles de un docente específico, haga clic en "Ver Detalles".
    \item Puede filtrar las alertas por período, programa o nivel de gravedad.
\end{enumerate}

\subsubsection{Seguimiento a Planes de Mejora}
\begin{enumerate}
    \item En el panel de decano, haga clic en "Seguimiento Plan de Mejora".
    \item Seleccione el acta de compromiso que desea revisar.
    \item Registre el avance del docente en cada compromiso establecido.
    \item Actualice el estado general del plan (En progreso, Completado, No cumplido).
    \item Haga clic en "Guardar Seguimiento".
\end{enumerate}

\subsubsection{Gestión de Procesos de Sanción}
\begin{enumerate}
    \item En el panel de decano, haga clic en "Proceso Sanción/Retiro".
    \item Para iniciar un nuevo proceso, haga clic en "Nuevo Proceso".
    \item Complete el formulario con:
    \begin{itemize}
        \item Datos del docente
        \item Tipo de sanción
        \item Motivo
        \item Evidencias
    \end{itemize}
    \item Haga clic en "Iniciar Proceso".
    \item Para actualizar un proceso existente, haga clic en "Actualizar" junto al proceso deseado.
    \item Puede filtrar los procesos por tipo de sanción o por calificación.
\end{enumerate}

\subsection{Funcionalidades para Docentes}
\subsubsection{Acceso al Panel de Docente}
\begin{enumerate}
    \item Inicie sesión con credenciales de docente.
    \item Será redirigido automáticamente al panel de docente.
    \item Alternativamente, puede acceder mediante la ruta: \texttt{/docente}
\end{enumerate}

\subsubsection{Consulta de Resultados de Evaluación}
\begin{enumerate}
    \item En el panel de docente, haga clic en "Resultados".
    \item Seleccione el período de evaluación que desea consultar.
    \item Visualizará los resultados desglosados por tipo de evaluación:
    \begin{itemize}
        \item Evaluación por estudiantes
        \item Evaluación administrativa
        \item Autoevaluación
    \end{itemize}
    \item Para ver detalles específicos, haga clic en "Ver Detalles" en cada sección.
\end{enumerate}

\subsubsection{Visualización de Estadísticas Personales}
\begin{enumerate}
    \item En el panel de docente, podrá ver gráficos estadísticos de su desempeño.
    \item Para comparar resultados entre diferentes períodos, seleccione los períodos deseados.
    \item Puede filtrar las estadísticas por tipo de evaluación o criterio específico.
\end{enumerate}

\begin{figure}[H]
  \centering
  \includegraphics[width=0.7\textwidth]{images/docente_estadisticas.png}
  \caption{Gráficos de desempeño personal (Docente)}
  \label{fig:docente_stats}
\end{figure}


\subsubsection{Configuración de Datos Personales}
\begin{enumerate}
    \item En el panel de docente, haga clic en "Configuración".
    \item Actualice sus datos personales según sea necesario.
    \item Para cambiar su contraseña, haga clic en "Cambiar Contraseña".
    \item Complete el formulario con su contraseña actual y la nueva contraseña.
    \item Haga clic en "Guardar Cambios".
\end{enumerate}

% 5. Información sobre comandos de software
\section{Mensajes del Sistema}\label{sec:mensajes}
\subsection{Mensajes de Error}
\begin{table}[H]
  \centering
  \caption{Mensajes de Error}\label{tab:mensajes-error}
  \begin{tabularx}{\textwidth}{>{\bfseries}lX}
    \toprule
    Mensaje & Descripción y Solución \\
    \midrule
    "Credenciales incorrectas" & Las credenciales proporcionadas no son válidas. Verifique su nombre de usuario y contraseña e intente nuevamente. \\
    "Acceso denegado" & No tiene permisos para acceder a la funcionalidad solicitada. Contacte al administrador. \\
    "Error de conexión a la base de datos" & Problema de conectividad con la base de datos. Intente nuevamente más tarde o contacte al soporte técnico. \\
    "Período no disponible" & El período seleccionado no está activo o ha expirado. Seleccione un período válido. \\
    \bottomrule
  \end{tabularx}
\end{table}

\subsection{Mensajes de Éxito}
\begin{table}[H]
  \centering
  \caption{Mensajes de Éxito}\label{tab:mensajes-exito}
  \begin{tabularx}{\textwidth}{>{\bfseries}lX}
    \toprule
    Mensaje & Descripción \\
    \midrule
    "Evaluación completada exitosamente" & La evaluación ha sido registrada correctamente en el sistema. \\
    "Acta de compromiso guardada" & El acta de compromiso ha sido creada y almacenada correctamente. \\
    "Proceso de sanción iniciado" & El proceso de sanción ha sido creado y registrado en el sistema. \\
    "Cambios guardados" & Los cambios realizados han sido guardados correctamente. \\
    \bottomrule
  \end{tabularx}
\end{table}

% 6. Procedimientos de error
\section{Solución de Problemas}\label{sec:solucion-problemas}
\subsection{Problemas Comunes y Soluciones}
\begin{tabularx}{\textwidth}{>{\bfseries}lX}
\toprule
Problema & Solución \\
\midrule
No puede acceder al sistema & Verifique su conexión a internet. Confirme que está utilizando las credenciales correctas. Si el problema persiste, contacte al administrador. \\
Los gráficos no se cargan & Actualice la página. Verifique que JavaScript esté habilitado en su navegador. Intente con otro navegador compatible. \\
Error al generar reportes & Verifique que haya seleccionado todos los filtros requeridos. Si el problema persiste, contacte al soporte técnico. \\
No visualiza todas las opciones en su panel & Verifique que tenga los permisos adecuados para su rol. Contacte al administrador para revisar sus permisos. \\
\bottomrule
\end{tabularx}

\\begin{figure}[H]
  \centering
  % Redimensiona todo el diagrama al ancho del texto
  \resizebox{\textwidth}{!}{%
    \begin{tikzpicture}[
        node distance=8mm and 15mm,
        decision/.style={diamond, draw, align=center, inner sep=1pt, text width=3cm},
        block/.style={rectangle, draw, rounded corners, align=center, inner sep=3pt, text width=3cm},
        arrow/.style={-{Latex[length=2.5mm]}, thick}
      ]
      % Nodos
      \node[block] (start) {¿Qué tipo de problema tienes?};
      \node[decision, below=of start] (tipo) {1) No accede \\ 2) Gráficos \\ 3) Reportes \\ 4) Opciones};

      % Rama 1
      \node[decision, below left=of tipo] (internet) {¿Conexión a Internet?};
      \node[block, below=of internet] (revisa) {Verifica tu conexión};
      \node[decision, below=of revisa] (cred) {¿Credenciales correctas?};
      \node[block, below left=of cred] (errorcred) {Revisa credenciales};
      \node[block, below right=of cred] (loginok) {Acceso exitoso};

      % Rama 2
      \node[decision, right=20mm of tipo] (js) {¿JavaScript habilitado?};
      \node[block, below=of js] (habilita) {Habilita JS y recarga};
      \node[block, below left=of habilita] (otro) {Prueba otro navegador};
      \node[block, below right=of habilita] (grafok) {Gráficos cargan};

      % Rama 3
      \node[decision, below right=of tipo] (filtros) {¿Filtros completos?};
      \node[block, below=of filtros] (completa) {Selecciona todos los filtros};
      \node[block, below left=of completa] (reintento) {Reintenta reporte};
      \node[block, below right=of completa] (soporte) {Contacta soporte};

      % Rama 4
      \node[decision, right=20mm of js] (perm) {¿Tienes permisos?};
      \node[block, below=of perm] (revisaP) {Revisa roles y permisos};
      \node[block, below left=of revisaP] (admin) {Contacta administrador};
      \node[block, below right=of revisaP] (okops) {Opciones disponibles};

      % Conexiones
      \draw[arrow] (start) -- (tipo);
      \draw[arrow] (tipo) -- node[left] {1} (internet);
      \draw[arrow] (internet) -- (revisa);
      \draw[arrow] (revisa) -- (cred);
      \draw[arrow] (cred) -- node[left] {No} (errorcred);
      \draw[arrow] (cred) -- node[right] {Sí} (loginok);

      \draw[arrow] (tipo) -- node[above] {2} (js);
      \draw[arrow] (js) -- (habilita);
      \draw[arrow] (habilita) -- node[below left] {No} (otro);
      \draw[arrow] (habilita) -- node[below right] {Sí} (grafok);

      \draw[arrow] (tipo) -- node[right] {3} (filtros);
      \draw[arrow] (filtros) -- (completa);
      \draw[arrow] (completa) -- node[left] {Intentar} (reintento);
      \draw[arrow] (completa) -- node[right] {Sí} (soporte);

      \draw[arrow] (tipo) ++(1cm,-1.2cm) |- node[above] {4} (perm);
      \draw[arrow] (perm) -- (revisaP);
      \draw[arrow] (revisaP) -- node[left] {No} (admin);
      \draw[arrow] (revisaP) -- node[right] {Sí} (okops);
    \end{tikzpicture}%
  }
  \caption{Árbol de decisión para solución de problemas comunes}
  \label{fig:flowchart}
\end{figure}



\subsection{Contacto de Soporte Técnico}
Si experimenta problemas que no puede resolver utilizando este manual, contacte al soporte técnico:

\begin{itemize}
    \item Correo electrónico: thomas.montoya.m@uniautonoma.edu.co
    \item Teléfono: 312-6081990
    \item Horario de atención: Lunes a Viernes, 9:00 AM - 6:00 PM
\end{itemize}

% 7. Glosario
\section{Glosario}\label{sec:glosario}
\begin{table}[H]
  \centering
  \caption{Glosario de Términos}\label{tab:glosario}
  \begin{tabularx}{\textwidth}{>{\bfseries}lX}
    \toprule
    Término & Definición \\
    \midrule
    Acta de Compromiso & Documento formal que registra los compromisos adquiridos por un docente para mejorar su desempeño. \\
    Dashboard & Panel de control personalizado que muestra información relevante según el rol del usuario. \\
    Evaluación Administrativa & Evaluación realizada por decanos o coordinadores sobre el desempeño de los docentes. \\
    Período de Evaluación & Intervalo de tiempo definido durante el cual se realizan las evaluaciones docentes. \\
    Plan de Mejora & Conjunto de acciones específicas que debe implementar un docente para mejorar su desempeño. \\
    Proceso de Sanción & Procedimiento disciplinario iniciado cuando un docente no cumple con los compromisos establecidos. \\
    \bottomrule
  \end{tabularx}
\end{table}

% 8. Notas
\section{Notas}\label{sec:notas}
\subsection{Recomendaciones de Uso}
\begin{itemize}
    \item Utilice navegadores actualizados como Chrome, Firefox, Safari o Edge para una mejor experiencia.
    \item Cambie su contraseña periódicamente para mantener la seguridad de su cuenta.
    \item Cierre sesión después de utilizar el sistema, especialmente en equipos compartidos.
    \item Verifique que los datos ingresados sean correctos antes de guardar cambios.
    \item Consulte regularmente las actualizaciones del sistema y los nuevos períodos de evaluación.
\end{itemize}



\subsection{Limitaciones Conocidas}
\begin{itemize}
    \item El sistema podría experimentar lentitud durante períodos de alta demanda (finales de semestre).
    \item Algunas funciones de exportación de reportes podrían no ser compatibles con versiones antiguas de Microsoft Excel.
    \item La visualización óptima está diseñada para pantallas con resolución mínima de 1024x768 píxeles.
\end{itemize}

% 9. Índice
\clearpage
\begin{thebibliography}{1}
  \addcontentsline{toc}{section}{Referencias}
  \bibitem{IEEE1063}
    IEEE Std 1063-2001, \textit{IEEE Standard for Software User Documentation}, IEEE, 2001.
\end{thebibliography}

% 10. Índice Alfabético
\clearpage
\section*{Índice Alfabético}
\addcontentsline{toc}{section}{Índice Alfabético}
\begin{tabularx}{\textwidth}{lX}
  \toprule
  \textbf{A} & \\
  Actas de Compromiso & \pageref{sec:glosario} \\
  Alertas de Bajo Desempeño & \pageref{sec:conceptos} \\
  Autoevaluación & \pageref{sec:procedimientos} \\
  \midrule
  \textbf{C} & \\
  Configuración & \pageref{sec:procedimientos} \\
  Credenciales & \pageref{sec:mensajes} \\
  \midrule
  \textbf{E} & \\
  Estadísticas & \pageref{sec:conceptos} \\
  Evaluación & \pageref{sec:introduccion} \\
  \midrule
  \textbf{P} & \\
  Períodos de Evaluación & \pageref{sec:procedimientos} \\
  Procesos de Sanción & \pageref{sec:procedimientos} \\
  \midrule
  \textbf{R} & \\
  Reportes & \pageref{sec:procedimientos} \\
  Roles & \pageref{sec:conceptos} \\
  \bottomrule
  
\end{tabularx}

\end{document}